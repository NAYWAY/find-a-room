%%%%%%%%%%%%%%%%%%%%%%%%%%%%%%%%%%%%%%%%%
% University/School Laboratory Report
% LaTeX Template
% Version 3.1 (25/3/14)
%
% This template has been downloaded from:
% http://www.LaTeXTemplates.com
%
% Original author:
% Linux and Unix Users Group at Virginia Tech Wiki 
% (https://vtluug.org/wiki/Example_LaTeX_chem_lab_report)
%
% License:
% CC BY-NC-SA 3.0 (http://creativecommons.org/licenses/by-nc-sa/3.0/)
%
%%%%%%%%%%%%%%%%%%%%%%%%%%%%%%%%%%%%%%%%%

%----------------------------------------------------------------------------------------
%	PACKAGES AND DOCUMENT CONFIGURATIONS
%----------------------------------------------------------------------------------------

\documentclass[12pt]{article}

%\usepackage[version=3]{mhchem} % Package for chemical equation typesetting
%\usepackage{siunitx} % Provides the \SI{}{} and \si{} command for typesetting SI units
\usepackage[left=1in,top=1in,right=1in,bottom=1in]{geometry} % Document margins
\usepackage{graphicx} % Required for the inclusion of images
\usepackage{pdfpages}
\usepackage{natbib} % Required to change bibliography style to APA
\usepackage{amsmath} % Required for some math elements 

\setlength\parindent{0pt} % Removes all indentation from paragraphs

\renewcommand{\labelenumi}{\alph{enumi}.} % Make numbering in the enumerate environment by letter rather than number (e.g. section 6)

%\usepackage{times} % Uncomment to use the Times New Roman font

%----------------------------------------------------------------------------------------
%	DOCUMENT INFORMATION
%----------------------------------------------------------------------------------------

\title{\textbf{Find A Room} \\ Design Document \\ CS 307} % Title

\author{Team \textsc{13}(Snoxy)} % Author name

\date{\today} % Date for the report

\begin{document}

\maketitle % Insert the title, author and date

\begin{center}
\begin{tabular}{l r}
Members: & Nathan Chang \\ % Partner names
& Xiaojing Ji \\
& Zilun Mai(Owen) \\
& \textbf{Saranyu Phusit(Team Leader)} \\
& Yao Xiao \\
\\
\bigskip
Instructor: & Professor Buster Dunsmore \\% Instructor/supervisor 
Project Coordinator: & Miguel Villarreal-Vasquez % Instructor/supervisor

\end{tabular}
\end{center}

\newpage
\tableofcontents

% If you wish to include an abstract, uncomment the lines below
% \begin{abstract}
% Abstract text
% \end{abstract}

%----------------------------------------------------------------------------------------
%	SECTION 1
%----------------------------------------------------------------------------------------


\newpage
\section{Purpose}

A mobile application that will give directions indoors by scanning QR codes put on walls throughout a building. The app will direct them towards their destination from where they are.


\section{Design Outline}

\subsection{Requirements}
The design should meet the following requirements.

\begin{itemize}
\item Having a visual guide with photos for navigating me to destination.
\item Navigating to the closest restrooms and water fountains and other destinations.
\item Getting the current location.
\item Automation adding my building map into an app and specify important destinations at ease. 
\item Response Time: The response time after user starts walking should not be more than 1 second

\end{itemize}

\subsection{General Priority}
The decisions that we make in this document are based on the priorities that we have set for the
project. These are (in order of importance):
\begin{itemize}
\item \textbf{Reliability:} The app should lead the user to the destination as quickly and precisely as possible. We take it seriously since that's the most important part of our project.
\item \textbf{Scalability:} The app should work with any building with basic structure.
\item \textbf{Size:} The size of the app should be small, the user should download it quickly.
\item \textbf{Supportability:} The app should work without having the developers add the map and checkpoints manually. We may add it if time allows.
\end{itemize}

\subsection{Outline Figures}


Figure here...



\section{Design Issue}

We run into several issues.

\subsection{Functional Issue}
\subsubsection{Issue 1}
Mobile app or Website app? \\
Option 1: Mobile \\
Option 2: Webapp \\

We choose option 2 since we can use XXX to transfer to a mobile program and it doesn't need native support for iOS/Andriod.
...

\subsection{Non-functional Issue}
\subsubsection{Issue X}
...

\section{Design Details}

\subsection{Components}

Diagram...
\subsection{Process}

\subsection{UI mockup}
\end{document}